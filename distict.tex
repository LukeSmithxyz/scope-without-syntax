\documentclass{article}

\title{Distinctness Proposal}
\author{Luke Smith}
\date{Fall 2017}

\begin{document}

\maketitle
	
\section{Abstract of Prelim 1}

\begin{abstract}
Here I argue that the concept of narrowly syntactic parameters is unnecessary, and unbefitting of a Minimalist model of the language faculty.
I attempt to describe an area of language classically thought of as being syntax-qua-syntax, that is, word order, and argue that the word order differences found in different languages can be said to be derived from differing prosodic constraints.
To implement this, I craft an Optimality Theoretic account of the canonical word order of sentential constituents (the subject, object and verb), which closely approximates the real-world typology of existing languages, all motivated by phonological principles already existing, or with close analogs in the literature.
That is, pre-existing prosodic constraints are sufficient to determine a language's word order.
I also show the enormous theoretical gains of this type of approach, noting the economy not just gained in theoretical simplicity, but in the clear account of how language is acquired by infants, that is, by a kind of robust phonological bootstrapping.
\end{abstract}

\section{Abstract of Prelim 2}

\begin{abstract}
	Here I argue that the commonly (and uncommonly) known facts about the availability of quantifier scope interpretations fall out cleanly from communicative constraints which Speakers and Hearers tactically navigate to converge on the intended meaning of an utterance.
	This allows a relatively complete and motivated theory of quantifier scope ambiguity wholly without the need to resort to syntactic structure \textit{per se} for the main data.
	I model this theory Game Theoretically, in a game where speakers receive a payoff for successful communication, and decrements to payoffs for the use of marked constructions.
	These assumptions are sufficient to account for classical scope ambiguity data, but also newly compiled data I present which argues that \emph{word order rigidity}, across languages and constructions is the cause of scope ambiguity.
\end{abstract}



\section{Explanation of Distinctness}

My first prelim is built principally on the data problem of differing word-orders across languages, the main analytical tool used being conventional Optimality Theory. The argument and intuition of the paper is that established prosodic constraints are sufficient for accounting for and motivating different word orders across languages, and also yiled a typology of word orders quite similar to that in the real world.

My second covers data of quantifier scope ambiguity and how it differs between an in different languages. For this I've used an analytical tool relatively unused in linguistics generally: that is Game Theory. The main goal is replacing inconsistent derivationally-based accounts of scope differences with a more plausible and motivated pragmatic model, built on speakers receiving payoffs or payoff reductions from effective communication or marked constructions respectively.

Besides the thematic commonality of desired theoretical economy, the data problems and tools in both papers are totally non-overlapping and dissimilar.

To sum up:

\begin{tabular}{r||cc}
	&Prelim 1&Prelim 2\\\hline\hline
	Data&Word order&Quantifier scope\\
	Subfield&Syntax or prosody&Pragmatics or semantics\\
	Tools&Optimality Theory&Game Theory\\
\end{tabular}

\end{document}

